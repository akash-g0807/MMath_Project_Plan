The reason I am doing this topic is because I find it interesting how algebraic structures such as monoids can be used to classify and understand formal languages and automata. 
Eilenberg's correspondence is a powerful tool that connects these two areas, allowing for a deeper understanding of the properties of regular languages through the lens of algebra.
For example star free languages correspond to aperiodic monoids.

This topic has both theoretical and computational aspects which I can explore which makes it appealing to me. Also after taking
a course on semigroup theory last year, I developed a fascination in the subject and would like to explore its applications in automata theory.

\section*{Issues I hope to address}
The are several issues I hope to address in this project:
\begin{itemize}
    \item {\bf Scope:} The scope of the project is quite large. It might take a lot to build up to Eilenberg's correspondence and the membership problem may be too ambitious to tackle in the time frame. 
    Nevertheless, even with Eilenberg's correspondence alone, I can gain a lot of insight into the classification of regular languages and could do a broad survey of membership without trying to implement algorithms.
    \item {\bf Computational challenges:} Implementing algorithms to test membership in pseudovarities may be too complex in the time frame I am given. But I still hope to tryt and compute syntactic monoids of small DFAs using GAP since that is a finite simple process.
    \item {\bf Timeframe I have set for the literature:} In the gantt chart below, I have outlined when I hope to what task. But it is based on my current understanding of the topic. As I read more literature, I may find that some topics take longer to understand than I initially thought.
    It may take longer also due to lectures, exams and other commitments. I will try to manage my time effectively to ensure I can cover the necessary material within the timeframe. 
 \end{itemize}

 But overall, I can definitely get to Eilenberg's correspondence and understand how varieties of regular languages and pseudovarieties of finite monoids are connected. And then just do a broad survey of membership of known pseudovarieties.

 In testing membership, Reiterman's theorem is useful. But I will be omitting the details of profinite identities in this project since the topological arguments are not in the scope of this project.