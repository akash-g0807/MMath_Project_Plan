I have already finished reading through the relevant lecture notes provided by my supervisor. These notes cover the key concepts and techniques in formal languages and automata theory.

The next step is to delve deeper into the topic of varieties of formal languages. I am currently studying a book titled ``Varieties of Formal Languages" by Jean-Éric Pin. This book provides a comprehensive overview of the theory of varieties and their applications in formal languages.
This text will be the main reference for understanding how regular languages can be classified using algebraic structures.

In addition to this, I plan to consult the following textbooks/research methods:
\begin{itemize}
    \item Samuel Eilenberg's ``Automata, Languages, and Machines" which provides foundational knowledge on automata theory and formal languages.
    \item John McKay's ``Automata and Languages" which is a gentle introduction to the subject.
    \item I will also look into research papers into the membership problem for pseudovarieties of finite monoids to understand the current state of research in this area.
\end{itemize}

Using literature, I will do 2 things:

\begin{enumerate}
    \item {\bf Theoretical study:} I will study the structure of the syntactic monoid of regular languages and understand how varieties and pseudovarieties are defined and classified. I will explore Eilenberg's correspondence in detail and understand how it can be used to classify regular languages based on the properties of their syntactic monoids.
    \item {\bf Computational study:} I will use computational tools such as GAP (Groups, Algorithms and Programming) to experiment and compute syntactic monoids of regular languages of small DFAs. I will try to implement algorithms to test membership in certain pseudovarieties of finite monoids and analyze their efficiency and effectiveness. Through this I want to shed light on the membership problem and see if it will yield new examples/insights.
\end{enumerate}

{\bf Where things might go wrong:}
\begin{itemize}
    \item Might be a bit too ambitious in terms of the scope of the project. The theoretical study might take longer than expected, leaving less time for the computational study.
    \item The computational study may face challenges such as the limitations of the GAP system or the complexity of the algorithms to implemented. This could result in incomplete or inaccurate findings.
\end{itemize}