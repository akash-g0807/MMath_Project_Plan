%%% Packages
\usepackage{amsmath,amsthm,amssymb,amsfonts}
\usepackage{graphicx}
\usepackage{xspace}
\usepackage{xcolor}
\usepackage[
	pdftitle=Project,
	colorlinks=false,
	final,
	linkcolor=blue,
	citecolor=magenta,
	urlcolor=blue,
]{hyperref}
\usepackage[obeyDraft, textsize=footnotesize, colorinlistoftodos]{todonotes}
\usepackage{verbatim}
\usepackage{enumitem}
\usepackage[toc]{appendix}
\usepackage{array}
\usepackage{cancel}
\usepackage{framed}

\setlist[enumerate,1]{label=(\arabic*)} % to make basic enumerate with 1) instead of 1.


\renewcommand{\thesection}{\arabic{section}}
\renewcommand{\thesubsection}{\thesection.\arabic{subsection}}
\renewcommand{\thesubsubsection}{\thesubsection.\arabic{subsubsection}}


\hbadness=10000

\usepackage{multicol}
\usepackage[
	paper=a4paper, % Change to letterpaper for US letter
	bottom=3cm, % Bottom margin
]{geometry} %% geomatry dash??

\usepackage{soul} %% like Aretha Franklin??
\usepackage{multicol}
\usepackage[skip=5pt plus1pt, indent=0pt]{parskip}
\usepackage{realhats}

\usepackage{multirow}
\usepackage{tikz-cd} % sonic cd???
\usetikzlibrary{graphs, 3d} % i put this here btw

%\usepackage[square,numbers]{natbib}%, sorting=nyt]
\usepackage[style=numeric, backend=biber, sorting=nyt]{biblatex}
\addbibresource{refs.bib}

\usepackage{float,caption}

%%%%% fancy hdr and sections
\usepackage{titlesec}
\titleformat{\chapter}
{\normalfont\huge\bfseries}{\chaptertitlename\ \thechapter.}{1em}{}

\usepackage{fancyhdr}
\pagestyle{fancy}
\fancyhf{}
\setlength{\headheight}{14pt}

\makeatletter
\renewcommand{\sectionmark}[1]{\markright{\thesection~~~#1}}
\renewcommand{\chaptermark}[1]{\markboth{\if@mainmatter \fi#1}{}}
\makeatother

\makeatletter
\newcommand{\chapterauthor}[1]{%
	{\parindent0pt\vspace*{-25pt}%
			\linespread{1.1}\Large\centering#1%
			\par\nobreak\vspace*{35pt}}
	\@afterheading%
}
\makeatother


% part title in each part
\fancyhead[L]{\nouppercase{\leftmark}}
\fancyhead[R]{Invariant Theory}
\cfoot{\thepage}

\newcommand{\Sn}[1][n]{\mathcal{S}_{#1}\xspace}
\newcommand{\An}[1][n]{\mathcal{A}_{#1}\xspace}
\newcommand{\set}[1]{\ensuremath{\left\lbrace #1 \right\rbrace}\xspace}

\newcommand\restr[2]{{% we make the whole thing an ordinary symbol
\left.\kern-\nulldelimiterspace % automatically resize the bar with \right
#1 % the function
\littletaller % pretend it's a little taller at normal size
\right|_{#2} % this is the delimiter
}}
\newcommand{\littletaller}{\mathchoice{\vphantom{\big|}}{}{}{}}


%%% ===== Styling for theorems
\theoremstyle{plain} % the style for theorem, propositions and lemmas
% Counter used to make it continuous numbering with subsubsection value appended
\newtheorem{thm}{Theorem}[section] % name shortcut for Theorem & counter check
\newtheorem{theorem}[thm]{Theorem}
\newtheorem{prop}[thm]{Proposition}
\newtheorem{proposition}[thm]{Proposition}
\newtheorem{lem}[thm]{Lemma}
\newtheorem{lemma}[thm]{Lemma}
\newtheorem{cor}[thm]{Corollary}
\newtheorem{corollary}[thm]{Corollary}
\newtheorem{conj}[thm]{Conjecture}
\newtheorem{claim}{Claim}
\newcounter{scount}[subsection] % counter to only go down to subsection but make it different
\renewcommand{\thescount}{\arabic{subsection}.\arabic{scount}}
\newtheorem{scholie}[scount]{Scholie}
\theoremstyle{definition} % the style for definitions
\newtheorem{defn}[thm]{Definition} % shortname
\newtheorem{definition}[thm]{Definition} % shortname
\newtheorem*{defns}{Definitions} % plural shortname
\newtheorem{ex}[thm]{Example}
\newtheorem{example}[thm]{Example}
\theoremstyle{remark} % the style for remarks and examples
\newtheorem{rem}[thm]{/emark} % shortname
\newtheorem{remark}[thm]{Remark} % longname
\newtheorem*{rem*}{Remark} % starred version
\newtheorem*{note}{Note}
\newtheorem*{rems}{Remarks} % plural
\newtheorem{notation}[thm]{Notation}
\newtheorem*{notation*}{Notation}
\newtheorem{question}[thm]{Question}
\counterwithin{equation}{section}
\counterwithin{figure}{section}

%%% Shorthands
\newcommand{\N}{\ensuremath{\mathbb{N}}}
\newcommand{\Z}{\ensuremath{\mathbb{Z}}}
\newcommand{\Q}{\ensuremath{\mathbb{Q}}}
\newcommand{\R}{\ensuremath{\mathbb{R}}}
\newcommand{\C}{\ensuremath{\mathbb{C}}}
\newcommand{\F}{\ensuremath{\mathbb{F}}}
\newcommand{\Card}[1]{\lvert #1 \rvert}
\newcommand{\priv}{\setminus}
\newcommand{\ob}[1]{\overline{#1}}
\newcommand{\ssi}{\Leftrightarrow}
\newcommand{\g}{\mathbf{g}}

%% Math Operators

\DeclareMathOperator{\GL}{GL}
\DeclareMathOperator{\Dim}{Dim}
\DeclareMathOperator{\Rank}{Rank}
\DeclareMathOperator{\Ham}{Ham}
\DeclareMathOperator{\Fix}{Fix}
\DeclareMathOperator{\Jac}{Jac}


%%% Todo notes
% Colouring
\def\mathcolor#1#{\@mathcolor{#1}}
\def\@mathcolor#1#2#3{%
	\protect\leavevmode
	\begingroup\color#1{#2}#3\endgroup
}
\newcommand{\blue}[1]{\mathcolor{blue}{#1}}
\newcommand{\rose}[1]{\mathcolor{magenta}{#1}}

% Todo items
\definecolor{cadetblue}{rgb}{0.37, 0.62, 0.63} % ocean blue
\definecolor{celadon}{rgb}{0.67, 0.88, 0.69} % light green
\definecolor{arylide}{rgb}{0.91, 0.84, 0.42} % mustardy yellow
\newcommand{\checkref}[2][]{\todo[color=cadetblue, #1]{#2}}
\newcommand{\suggestion}[2][]{\todo[color=celadon, inline, #1]{#2}}
\newcommand{\idea}[2][]{\todo[color=magenta!70!black, #1]{#2}}
\newcommand{\error}[2][]{\todo[color=red!70, #1]{#2}}


%% If you want to add packages add them under THIS LINE
%% They will get edited later
